\chapter{Parametric Distributions}

\section{Test Description}
\subsection{Test Coverage}
The tables in the chapter present the results generated by the following
functions exported by the DLL:
\begin{itemize}
\item  \texttt{Java\_redcas\_numerical\_Native[Distribution]\_getMean()}
\item  \texttt{Java\_redcas\_numerical\_Native[Distribution]\_getVariance()}
\item  \texttt{Java\_redcas\_numerical\_Native[Distribution]\_getQuantile()}
\item  \texttt{Java\_redcas\_numerical\_Native[Distribution]\_getPDF()}
\item  \texttt{Java\_redcas\_numerical\_Native[Distribution]\_getCDF()}
\item  \texttt{Java\_redcas\_numerical\_Native[Distribution]\_getSample()}
\end{itemize}
The following distributions are covered:

$\begin{array}{cc}
\text{Distribution} & \text{pdf}(x) \\ 
\text{Gamma}(\alpha ,\beta ) & \frac{1}{\beta ^{\alpha }\Gamma (\alpha )}x^{\alpha -1}e^{-x/\beta } \\ 
\text{Beta}(\alpha ,\beta ) & \frac{\Gamma (\alpha +\beta )}{\Gamma (\alpha)\Gamma (\beta )}x^{\alpha -1}(1-x)^{\beta -1} \\ 
\text{Normal}(\nu ,\sigma ) & \frac{1}{\sqrt{2\pi }\sigma }e^{-(\ln x-\mu)^{2}/2\sigma ^{2}} \\ 
\text{Lognormal}(\nu ,\tau ) & \frac{1}{\sqrt{2\pi }\tau x}e^{-(x-\nu)^{2}/2\tau ^{2}} \\ 
\text{Trunc.Lognormal}(\nu ,\tau ) & \frac{Lognormal(\nu ,\tau )}{\int_{x=0}^{1}Lognormal(\nu ,\tau )\cdot dx}
\end{array}$

Note that within the DLL a confusion exists regarding the parameters of the 
Gamma distribution. 

\subsection{Test Execution}
Test cases are executed in a two-stage manner: in the first stage, the test
cases are run without any external data files. This produces a file
`data.dat' which contains the Mathematica expressions corresponding to the
calls made to the above mentioned functions. Results of the evaluation of
these expressions in Mathematica are exported to a file `rslt.dat'. The test
cases are then run for a second time, which results in Mathematica results
being incorporated in the tables showing the results.

\subsection{Interpretation of Test Results}
Each subsection contains the test result for the parametric distribution 
indicated in the title of that subsection.

The structure of the first table in each section, of which an example is shown 
below, is as
follows. The top part shows the parameter values for the test case, plus the
computed and expected values of mean and variance. The bottom part shows the
computed and expected mean, PDF, and CDF values. The first column (`x')
lists the percentile values computed for the parametric distribution,
whereas column 2 (`exp') lists the expected values. Column 3 contains the
CDF\ values computed as a function of the value in the first column. Column
4 contains the expected CDF value. Finally, column 5 contains the PDF values
computed as a function of the value in the first column. Column 6 contains
the expected PDF value.

\begin{center}\small
\begin{tabular}{|r|r|r|r|r|r|}
\hline
\multicolumn{2}{|l|}{\lbrack Parameter 1 Name]} & \multicolumn{4}{|r|}{[%
\texttt{Parameter 1 Value]}} \\ 
\multicolumn{2}{|l|}{\lbrack Parameter 2 Name]} & \multicolumn{4}{|r|}{[%
\texttt{Parameter 2 Value]}} \\ 
\multicolumn{2}{|l|}{mean} & \multicolumn{4}{|r|}{\texttt{[computed mean]}}
\\ 
\multicolumn{2}{|l|}{exp. mean} & \multicolumn{4}{|r|}{\texttt{[expected
mean]}} \\ 
\multicolumn{2}{|l|}{variance} & \multicolumn{4}{|r|}{\texttt{[computed
variance]}} \\ 
\multicolumn{2}{|l|}{exp. variance} & \multicolumn{4}{|r|}{\texttt{[expected
variance]}} \\ \hline
$x$ & exp & CDF & exp & PDF & exp \\ \hline
&  &  &  &  &  \\ \hline
\end{tabular}\normalsize\end{center}

The second table in each section presents the validation results for the 
parametric distribution deviate generators, in terms of the percentile 
values obtained from the generated samples.

For each distribution, 60 sample sets are obtained, which are sorted and then
used to generate percentile values. The sample sets consist of 20 sets of 2,000
samples, 20 sets of 10,000 samples, and 20 sets of 100,000 samples. The sets 
are used to compute both mean and standard deviation for the generated percentile
values, grouped by sample size.

The table containing the sample results has the following columns
\begin{center}\begin{tabular}{|l|l|}\hline
$p$ & percentile \\
$x$ & expected percentile value \\
$x_{2K}$ & mean of return value (2,000 samples) \\
$sd_{2K}$ & standard deviation (2,000 samples) \\
$x_{10K}$ & mean of return value (10,000 samples) \\
$sd_{10K}$ & standard deviation (10,000 samples) \\
$x_{100K}$ & mean of return value (100,000 samples) \\
$sd_{100K}$ & standard deviation (100,000 samples) \\ \hline
\end{tabular}\end{center}
